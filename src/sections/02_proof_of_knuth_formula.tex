\begin{proposition}[Knuth's formula for Multifold sums of odd powers]
    \label{prop:knuth-sums-of-odd-powers}
    \begin{align*}
        \KnuthRFoldSum{r}{n}{2m-1} &= \sum_{k=1}^{m} (2k-1)! \binom{n+k-1+r}{2k-1+r} T(2m, 2k)
    \end{align*}
    where $T(n, k)$ is central factorial number $T(n, k) = \frac{1}{k!} \delta^k 0^n$ and $\delta^k 0^n$
    is the central finite difference of power in zero.
    \begin{proof}
        Consider the Riordan's power identity, see [cite]
        \begin{lemma} [Riordan's power identity]
            \begin{align*}
                n^m = \sum_{k=1}^{m} T(m,k) \centralFactorial{n}{k}
            \end{align*}
            where $\centralFactorial{n}{k}$ is central factorial [eq ref].
        \end{lemma}
        It is easy to see that Riordan's power identity is a direct consequence of Newton's series for central difference [eq ref],
        with $T(m,k)=\frac{1}{k!} \delta^k 0^m$, where $\delta^k 0^m$ is central difference of power in zero.
        Now we notice that $T(m,k) =0$ whether $m-k$ is odd.
        Thus,
        \begin{align*}
            n^{2m} = \sum_{k=1}^{2m} T(2m,k) \centralFactorial{n}{k}
        \end{align*}
        We allow $k$ to run over the integers $k=2,4,6,\ldots, 2m$ to maintain the condition $T(2m,k) \neq 0$,
        hence
        \begin{align*}
            n^{2m} = \sum_{k=1}^{m} T(2m,2k) \centralFactorial{n}{2k}
        \end{align*}
        By expressing the central factorials $\centralFactorial{n}{2k}$ in terms of falling factorials
        \begin{align*}
            \centralFactorial{n}{2k}
            = n \fallingFactorial{n+k-1}{2k-1}
        \end{align*}
        Yields
        \begin{align*}
            n^{2m} = \sum_{k=1}^{m} T(2m,2k) n \fallingFactorial{n+k-1}{2k-1}
        \end{align*}
        By dividing by $n$ an applying the identity $\frac{\fallingFactorial{x}{n}}{n!} = \binom{x}{n}$, we get
        \begin{align*}
            n^{2m-1} = \sum_{k=1}^{m} (2k-1)! \binom{n+k-1}{2k-1} T(2m,2k)
        \end{align*}
        which is the base identity for odd powers.

        Now, the ordinary sum of odd powers is
        \begin{align*}
            \KnuthRFoldSum{1}{n}{2m-1} = \sum_{k=1}^{m} (2k-1)! T(2m,2k) \sum_{j=1}^{n} \binom{j+k-1}{2k-1}
        \end{align*}
        By hockey-stick identity $\sum_{j=1}^{n} \binom{j+k-1}{2k-1} = \binom{n+k}{2k}$, thus
        \begin{align*}
            \KnuthRFoldSum{1}{n}{2m-1} = \sum_{k=1}^{m} (2k-1)! T(2m,2k) \binom{n+k}{2k}
        \end{align*}
        By induction yields
        \begin{align*}
            \KnuthRFoldSum{r}{n}{2m-1} &= \sum_{k=1}^{m} (2k-1)! \binom{n+k-1+r}{2k-1+r} T(2m, 2k)
        \end{align*}
    \end{proof}
\end{proposition}
We can compare the Knuth's formula and special case of [eqref] for $m \rightarrow 2m-1$, that is
\begin{corollary}
    \begin{align*}
        \KnuthRFoldSum{r}{n}{2m-1} = \sum_{k=0}^{m} k! \binom{n+\frac{k+1}{2}-1+r}{k+r} T(2m, 2k)
    \end{align*}
\end{corollary}
