In this manuscript we derive formula for multifold sums of powers
using Newton's formula and central differences.

The idea to derive sums of powers using difference operator and Newton's series is quite generic,
thus, formulas for sums of powers using forward and backward differences can be found
in the works~\cite{kolosov_2026_18135998, kolosov_2026_18136163}.

We define the following recurrence for multifold sums of powers, which we utilize throughout the paper.
\begin{align*}
    \KnuthRFoldSum{0}{n}{m}   &= n^m \\
    \KnuthRFoldSum{1}{n}{m}   &= \KnuthRFoldSum{0}{1}{m} + \KnuthRFoldSum{0}{2}{m} + \cdots + \KnuthRFoldSum{0}{n}{m} \\
    \KnuthRFoldSum{r+1}{n}{m} &= \KnuthRFoldSum{r}{1}{m} + \KnuthRFoldSum{r}{2}{m} + \cdots + \KnuthRFoldSum{r}{n}{m}
\end{align*}
Consider the Newton's formula~\cite{newton1850newton} in central differences
\begin{proposition}[Newton's series in central differences]
    \label{prop:newtons-series-in-central-differences}
    \begin{align*}
        f(x) &= \sum_{k=0}^{\infty} \frac{\centralFactorial{x}{k}}{k!} \delta^{k} f(0)
    \end{align*}
    where $\delta^{k} f(0)= \sum_{j=0}^{k} (-1)^{j} \binom{k}{j} f\left(\frac{k}{2} - j\right)$
    is central finite difference in zero, and $\centralFactorial{x}{k} = n \left( n + \frac{k}{2} -1 \right)\left( n + \frac{k}{2} -2 \right) \cdots \left( n - \frac{k}{2} +1 \right)$ is central factorial.
\end{proposition}

\begin{lemma}[Central factorial]
    \begin{align*}
        \centralFactorial{n}{k}
        = n \left( n + \frac{k}{2} -1 \right)\left( n + \frac{k}{2} -2 \right) \cdots \left( n - \frac{k}{2} +1 \right)
        = n \prod_{j=1}^{k-1} \left( n + \frac{k}{2} -j \right)
    \end{align*}
\end{lemma}
We observe that central factorials are closely related to falling factorials $\fallingFactorial{x}{n} = x(x-1)(x-2)(x-3)\cdots(x-n+1)=\prod_{k=0}^{n-1}(x-k)$.
Therefore,
\begin{align*}
    \centralFactorial{n}{k} = n \fallingFactorial{n+\frac{k}{2}-1}{k-1}
\end{align*}
To derive formula for multifold sums of powers, we follow the strategy to express the Newton's formula [eq ref] in terms
of binomial coefficients, then to reach closed forms of column sum of binomial coefficients by means of hockey stick identity.
Therefore,
\begin{proposition}
    For $k \geq 1$
    \begin{align*}
        \frac{\centralFactorial{n}{k}}{k!}
        &= \frac{n}{k!}\fallingFactorial{n+\frac{k}{2}-1}{k-1}
        = \frac{n}{k (k-1)!} \fallingFactorial{n+\frac{k}{2}-1}{k-1}
        = \frac{n}{k} \binom{n+\frac{k}{2}-1}{k-1}
    \end{align*}
    \begin{proof}
        The identity above is true because $\frac{\fallingFactorial{x}{n}}{n!} = \binom{x}{n}$.
    \end{proof}
\end{proposition}

Which yields Newton's formula for powers, in terms of central differences.
For positive integers $n \geq 1$ and $m \geq 1$
\begin{align*}
    n^m = \sum_{k=1}^{m} \frac{n}{k} \binom{n+\frac{k}{2}-1}{k-1} \delta^{k} 0^m
\end{align*}

Although based on Newton's interpolation series, the formula above
begins the summation from $k=1$ to avoid division by zero in $\frac{n}{k}$.
It is a valid trick, because the central difference $\delta^{k} 0^n$ is zero for all $n \geq 1$ and $k=0$.

By factoring out and simplifying the term $n$, we get
\begin{align*}
    n^{m-1} = \sum_{k=1}^{m} \frac{1}{k} \binom{n+\frac{k}{2}-1}{k-1} \delta^{k} 0^m
\end{align*}
We may observe that the operator of central finite difference of power $\delta^{k} 0^m$ requires the parity of
its arguments $m$ and $k$ meaning that both $m$ and $k$ must be $m\bmod2 = k\bmod2$ so that
$\delta^{k} 0^m$ is non-zero.

By setting $m \rightarrow 2m$ we get
\begin{align*}
    n^{2m-1} = \sum_{k=1}^{2m} \frac{1}{k} \binom{n+\frac{k}{2}-1}{k-1} \delta^{k} 0^{2m}
\end{align*}
Thus, the central difference $\delta^{k} 0^{2m}$ for odd $k$ is zero.
Since $k$ runs over all integers in the range $0\leq k \leq 2m$,
we are able to omit odd values of $k$
\begin{align*}
    n^{2m-1} = \sum_{k=1}^{m} \frac{1}{2k} \binom{n+k-1}{2k-1} \delta^{2k} 0^{2m}
\end{align*}
Thus, formula for ordinary sums of odd powers is
\begin{proposition}[Ordinary sums of powers]
    \begin{align*}
        \KnuthRFoldSum{1}{n}{2m-1} = \sum_{k=1}^{m} \frac{1}{2k} \binom{n+k}{2k} \delta^{2k} 0^{2m}
    \end{align*}
    \begin{proof}
        We have
        $\KnuthRFoldSum{1}{n}{2m-1} = \sum_{k=1}^{m} \frac{1}{2k} \delta^{2k} 0^{2m} \sum_{j=1}^{n} \binom{j+k-1}{k-1}$.

        By hockey stick identity $\sum_{j=1}^{n} \binom{j+k-1}{2k-1} = \binom{n+k}{2k}$, thus the statement follows.
    \end{proof}
\end{proposition}
Therefore,

\begin{theorem}[Multifold sums of odd powers]
    \begin{mdframed}
        \begin{align*}
            \KnuthRFoldSum{r}{n}{2m-1} = \sum_{k=1}^{m} \frac{1}{2k} \binom{n+k-1+r}{2k-1+r} \delta^{2k} 0^{2m}.
        \end{align*}
    \end{mdframed}
    \begin{proof}
        We have
        $\KnuthRFoldSum{1}{n}{2m-1} = \sum_{k=1}^{m} \frac{1}{2k} \delta^{2k} 0^{2m} \sum_{j=1}^{n} \binom{j+k-1}{2k-1}$.

        By hockey stick identity $\sum_{j=1}^{n} \binom{j+k-1}{k-1} = \binom{n+k}{2k}$.
        By induction the claim follows.
    \end{proof}
\end{theorem}

The formula for multifold sums of odd powers can be rewritten in terms of central factorial numbers of the second kind $T(n,k)$,
because
\begin{align*}
    k! T(n, k) = \delta^k 0^n
\end{align*}
The identity above is well discussed in~\cite{steffensen1927interpolation, carlitz1963divided, riordan1968combinatorial}.
The non-zero central factorial numbers $T(2m,2k)$ are registered as \href{https://oeis.org/A008957}{\texttt{A008957}} in the OEIS~\cite{sloane2003line}.

Hence,
\begin{proposition}[Multifold sums of odd powers in central factorial numbers]
    \begin{mdframed}
        \begin{align*}
            \KnuthRFoldSum{r}{n}{2m-1} = \sum_{k=1}^{m} (2k-1)! \binom{n+k-1+r}{2k-1+r} T(2m,2k).
        \end{align*}
    \end{mdframed}
\end{proposition}
It is remarkable that exactly this formula can be found in Donald Knuth's work
\textit{Johann Faulhaber and sums of powers}, see~\cite{knuth1993johann}.
