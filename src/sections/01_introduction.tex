\begin{proposition}[Newton's series in central differences]
    \label{prop:newtons-series-in-central-differences}
    \begin{align*}
        f(x) &= \sum_{k=0}^{\infty} \frac{\centralFactorial{x}{k}}{k!} \delta^{k} f(0)
    \end{align*}
    where $\delta^{k} f(0)= \sum_{j=0}^{k} (-1)^{j} \binom{k}{j} f\left(\frac{k}{2} - j\right)$
    is central finite difference in zero, and $\centralFactorial{x}{k} = n \left( n + \frac{k}{2} -1 \right)\left( n + \frac{k}{2} -2 \right) \cdots \left( n - \frac{k}{2} +1 \right)$ is central factorial.
\end{proposition}

\begin{lemma}[Central factorial]
    \begin{align*}
        \centralFactorial{n}{k}
        = n \left( n + \frac{k}{2} -1 \right)\left( n + \frac{k}{2} -2 \right) \cdots \left( n - \frac{k}{2} +1 \right)
        = n \prod_{j=1}^{k-1} \left( n + \frac{k}{2} -j \right)
    \end{align*}
\end{lemma}
We observe that central factorials are closely related to falling factorials $\fallingFactorial{x}{n} = x(x-1)(x-2)(x-3)\cdots(x-n+1)=\prod_{k=0}^{n-1}(x-k)$.
Therefore,
\begin{align*}
    \centralFactorial{n}{k} = n \fallingFactorial{n+\frac{k}{2}-1}{k-1}
\end{align*}
To derive formula for multifold sums of powers, we follow the strategy to express the Newton's formula [eq ref] in terms
of binomial coefficients, then to reach closed forms of column sum of binomial coefficients by means of hockey stick pattern.
Therefore,
\begin{proposition}
    For $k \geq 1$
    \begin{align*}
        \frac{\centralFactorial{n}{k}}{k!}
        &= \frac{n}{k!}\fallingFactorial{n+\frac{k}{2}-1}{k-1}
        = \frac{n}{k (k-1)!} \fallingFactorial{n+\frac{k}{2}-1}{k-1}
        = \frac{n}{k} \binom{n+\frac{k}{2}-1}{k-1}
    \end{align*}
    \begin{proof}
        The identity above is true because $\frac{\fallingFactorial{x}{n}}{n!} = \binom{x}{n}$.
    \end{proof}
\end{proposition}

Which yields Newton's formula for powers, in terms of central differences.
For positive integers $n \geq 1$ and $m \geq 1$
\begin{align*}
    n^m = \sum_{k=1}^{m} \frac{n}{k} \binom{n+\frac{k}{2}-1}{k-1} \delta^{k} 0^m
\end{align*}

In the proposition above,
we start the summation from $k=1$ to avoid division by zero in $\frac{n}{k}$.
It is a valid trick, because the central difference $\delta^{k} 0^n$ is zero for all $n \geq 1$ and $k=0$.

By factoring out and simplifying the term $n$, we get
\begin{align*}
    n^{m-1} = \sum_{k=1}^{m} \frac{1}{k} \binom{n+\frac{k}{2}-1}{k-1} \delta^{k} 0^m
\end{align*}
By reindexing the sum yields
\begin{proposition} [Newton's series for power in zero]
    \begin{align*}
        n^{m} = \sum_{k=0}^{m+1} \frac{1}{k+1} \binom{n+\frac{k+1}{2}-1}{k} \delta^{k+1} 0^{m+1}
    \end{align*}
\end{proposition}
Thus, formula for ordinary sums of powers follows
\begin{proposition} [Ordinary sums of powers]
    \begin{align*}
        \KnuthRFoldSum{1}{n}{m} = \sum_{k=0}^{m+1} \frac{1}{k+1} \binom{n+\frac{k+1}{2}}{k+1} \delta^{k+1} 0^{m+1}
    \end{align*}
    \begin{proof}
        We have
        $\KnuthRFoldSum{1}{n}{m} = \sum_{k=0}^{m} \frac{1}{k+1} \delta^{k+1} 0^{m+1} \sum_{j=1}^{n} \binom{j+\frac{k+1}{2}-1}{k}$.
        By hockey stick identity $\sum_{j=1}^{n} \binom{j+\frac{k+1}{2}-1}{k} = \binom{n+\frac{k+1}{2}}{k+1}$.
        Thus, the claim follows.
    \end{proof}
\end{proposition}
Continuing similarly, we get formula for multifold sums of powers
\begin{theorem}[Multifold sums of powers]
    \begin{mdframed}
        \begin{align*}
            \KnuthRFoldSum{r}{n}{m} = \sum_{k=0}^{m+1} \frac{1}{k+1} \binom{n+\frac{k+1}{2}-1+r}{k+r} \delta^{k+1} 0^{m+1}
        \end{align*}
    \end{mdframed}
\end{theorem}
Additionally, the formula for multifold sums of powers can be expressed in terms of central factorial numbers of the second kind [references].
In Riordan notation,

\begin{lemma} [Central factorial numbers]
    \begin{align*}
        T(n, k) = \frac{\delta^k 0^n}{k!}
    \end{align*}
\end{lemma}
Note that central factorial numbers of the second kind $T(n,k)$ are non-zero
only for pairs $(n,k)$ such that $n-k$ is even.
Meaning that $T(2n,2k)$ is always non-zero.
The triangle of central factorial numbers $T(2n,2k)$ is the sequence [ID] in the OEIS~\cite{sloane2003line}.

\begin{proposition}[Multifold sums of powers via central factorial numbers]
    \begin{mdframed}
        \begin{align*}
            \KnuthRFoldSum{r}{n}{m} = \sum_{k=0}^{m} k! \binom{n+\frac{k+1}{2}-1+r}{k+r} T(m+1, k+1)
        \end{align*}
    \end{mdframed}
\end{proposition}

