In this manuscript we derive formula for multifold sums of powers
using Newton's formula and central differences.

The idea to derive sums of powers using difference operator and Newton's series
is quite generic.
Thus, formulas for sums of powers using forward and backward differences
can be found in the works~\cite{kolosov_2026_18135998, kolosov_2026_18136163}.

We define the recurrence for multifold sums of powers introduced
by Donald Knuth in~\cite{knuth1993johann},
which we utilize throughout the paper.

\begin{proposition} [Multifold sums of powers recurrence]
  For non-negative integers $r,n,m$
  \begin{align*}
    \KnuthRFoldSum{0}{n}{m}   &= n^m \\
    \KnuthRFoldSum{1}{n}{m}   &= \KnuthRFoldSum{0}{1}{m}
    + \KnuthRFoldSum{0}{2}{m} + \cdots + \KnuthRFoldSum{0}{n}{m} \\
    \KnuthRFoldSum{r+1}{n}{m} &= \KnuthRFoldSum{r}{1}{m}
    + \KnuthRFoldSum{r}{2}{m} + \cdots + \KnuthRFoldSum{r}{n}{m}
  \end{align*}
\end{proposition}

\begin{proposition}[Central factorials]
  For integers $n,k$
  \begin{align*}
    \centralFactorial{n}{k} =
    \begin{cases}
      0, \quad & \mathrm{if \; } k < 0 \\
      1, \quad & \mathrm{if \; } k = 0 \\
      n \left( n + \frac{k}{2} -1 \right)\left( n + \frac{k}{2} -2 \right)
      \cdots \left( n - \frac{k}{2} +1 \right) = n \prod_{j=1}^{k-1}
      \left( n + \frac{k}{2} -j \right), \quad & \mathrm{if \; } k>0
    \end{cases}
  \end{align*}
\end{proposition}

Consider Newton's interpolation
formula~\cite{newton1850newton,butzer1989central}
in central differences evaluated in zero

\begin{proposition}[Newton's formula in central differences in zero]
  \label{prop:newtons-formula-in-central-differences}
  \begin{align*}
    f(x) &= \sum_{k=0}^{\infty} \frac{\centralFactorial{x}{k}}{k!} \delta^{k} f(0)
  \end{align*}
  where
  $\delta^{k} f(0)
  = \sum_{j=0}^{k} (-1)^{j} \binom{k}{j} f\left(\frac{k}{2} - j\right)$
  are central finite differences in zero,
  and $\centralFactorial{x}{k}$
  are central factorials,
  with $\centralFactorial{x}{0}=1$ for every $x$.
\end{proposition}

We observe that central factorials are closely related to falling factorials
$\fallingFactorial{x}{n}
= x(x-1)(x-2)(x-3)\cdots(x-n+1)=\prod_{k=0}^{n-1}(x-k)$.
Therefore,

\begin{proposition}[Central factorials in terms of falling]
  \label{prop:central-factorials-in-terms-of-falling}
  For integers $n,k$
  \begin{align*}
    \centralFactorial{n}{k} =
    \begin{cases}
      0, \quad & \mathrm{if \; } k<0 \\
      1, \quad & \mathrm{if \; } k=0 \\
      n \fallingFactorial{n+\frac{k}{2}-1}{k-1}, \quad & \mathrm{if \; } k>0
    \end{cases}
  \end{align*}
  where $\fallingFactorial{n+\frac{k}{2}-1}{k-1}$ are falling factorials.
\end{proposition}

To derive formula for multifold sums of powers,
we follow the strategy to express the Newton's
formula~\eqref{prop:newtons-formula-in-central-differences}
in terms of binomial coefficients,
then to reach closed forms of column sum of binomial coefficients
by means of hockey stick identity.
Therefore,

\begin{proposition}[Binomial form of central factorials]
  \label{prop:binomial-form-of-central-factorials}
  \begin{mdframed}
    For integers $n$ and $k \geq 1$
    \begin{align*}
      \frac{\centralFactorial{n}{k}}{k!}
      = \frac{n}{k} \binom{n+\frac{k}{2}-1}{k-1}
    \end{align*}
  \end{mdframed}
  \begin{proof}
    We have
    \begin{align*}
      \frac{\centralFactorial{n}{k}}{k!}
      &= \frac{n}{k!}\fallingFactorial{n+\frac{k}{2}-1}{k-1}
      = \frac{n}{k (k-1)!} \fallingFactorial{n+\frac{k}{2}-1}{k-1}
      = \frac{n}{k} \binom{n+\frac{k}{2}-1}{k-1}
    \end{align*}
    because of the identity in falling factorial
    $\frac{\fallingFactorial{x}{n}}{n!} = \binom{x}{n}$
    and~\eqref{prop:central-factorials-in-terms-of-falling}.
  \end{proof}
\end{proposition}

Which yields Newton's formula for powers, in terms of central differences.
\begin{proposition} [Newton's formula for powers in zero]
  \label{prop:newtons-formula-for-powers-in-zero}
  \begin{mdframed}
    For positive integers $n \geq 1$ and $m \geq 1$
    \begin{align*}
      n^m
      = \sum_{k=1}^{m} \frac{n}{k} \binom{n+\frac{k}{2}-1}{k-1} \delta^{k} 0^m
    \end{align*}
  \end{mdframed}
\end{proposition}
Although based on Newton's interpolation
formula~\eqref{prop:newtons-formula-in-central-differences},
the proposition~\eqref{prop:newtons-formula-for-powers-in-zero}
iterates starting fom $k=1$
to avoid division by zero in $\frac{n}{k}$.
This is a valid trick, because the central difference $\delta^{k} 0^n$
is zero for all $n \geq 1$ and $k=0$.

By factoring out and simplifying the term $n$, we get
\begin{align*}
  n^{m-1}
  = \sum_{k=1}^{m} \frac{1}{k} \binom{n+\frac{k}{2}-1}{k-1} \delta^{k} 0^m
\end{align*}
We may observe that the operator of central finite difference $\delta^{k} 0^m$
requires the parity of its arguments $m$ and $k$
meaning that both $m$ and $k$ required to be: $m\pmod{2} = k\pmod{2}$,
such that finite differences $\delta^{k} 0^m$ are non-zero
\begin{align*}
  \delta^{k} 0^m \neq 0,
  \quad &\mathrm{whether} \quad m\pmod{2} = k\pmod{2}, \\
  \delta^{k} 0^m = 0,
  \quad &\mathrm{whether} \quad m\pmod{2} \neq k\pmod{2}.
\end{align*}

By setting $m \rightarrow 2m$ we get
\begin{align*}
  n^{2m-1}
  = \sum_{k=1}^{2m} \frac{1}{k} \binom{n+\frac{k}{2}-1}{k-1} \delta^{k} 0^{2m}
\end{align*}
Thus, the central differences $\delta^{k} 0^{2m}$ are zero for all odd $k$.

Since that $k$ runs over all integers in the range $0\leq k \leq 2m$,
we can omit odd values of $k$
\begin{align*}
  n^{2m-1}
  = \sum_{k=1}^{m} \frac{1}{2k} \binom{n+k-1}{2k-1} \delta^{2k} 0^{2m}
\end{align*}
Hence, formula for ordinary sums of odd powers yields

\begin{proposition}[Ordinary sums of odd powers in central differences]
  \label{prop:ordinary-sums-of-odd-powers-in-central-differences}
  \begin{mdframed}
    For integers $n\geq 1, \; m\geq 1$
    \begin{align*}
      \KnuthRFoldSum{1}{n}{2m-1}
      = \sum_{k=1}^{m} \frac{1}{2k} \binom{n+k}{2k} \delta^{2k} 0^{2m}
    \end{align*}
  \end{mdframed}
  \begin{proof}
    We have
    $\KnuthRFoldSum{1}{n}{2m-1}
    = \sum_{k=1}^{m}
    \frac{1}{2k} \delta^{2k} 0^{2m} \sum_{j=1}^{n} \binom{j+k-1}{k-1}$.

    By hockey stick identity
    $\sum_{j=1}^{n} \binom{j+k-1}{2k-1} = \binom{n+k}{2k}$,
    thus the statement follows.
  \end{proof}
\end{proposition}

Therefore,

\begin{theorem}[Multifold sums of odd powers in central differences]
  \label{theorem:multifold-sums-of-odd-powers-in-central-differences}
  \begin{mdframed}
    For integers $n \geq 1, \; m \geq 1$ and $r\geq0$
    \begin{align*}
      \KnuthRFoldSum{r}{n}{2m-1}
      = \sum_{k=1}^{m} \frac{1}{2k} \binom{n+k-1+r}{2k-1+r} \delta^{2k} 0^{2m}.
    \end{align*}
  \end{mdframed}
  \begin{proof}
    We have
    $\KnuthRFoldSum{1}{n}{2m-1}
    = \sum_{k=1}^{m} \frac{1}{2k} \delta^{2k} 0^{2m}
    \sum_{j=1}^{n} \binom{j+k-1}{2k-1}$.

    By hockey stick identity $\sum_{j=1}^{n} \binom{j+k-1}{k-1} = \binom{n+k}{2k}$.
    By induction the claim follows.
  \end{proof}
\end{theorem}

It is quite interesting to notice that
theorem~\eqref{theorem:multifold-sums-of-odd-powers-in-central-differences}
is a central difference form of the formula of sums of odd-powers
given by Donald Knuth in
\textit{Johann Faulhaber and sums of powers},
see~\cite{knuth1993johann}.

The reason is straightforward, instead of using Central factorial numbers
of the second kind $T(n,k)$,
the theorem~\eqref{theorem:multifold-sums-of-odd-powers-in-central-differences}
utilizes central differences explicitly,
because

\begin{lemma} [Central factorial numbers of the second kind]
  \label{lem:central-factorial-numbers-of-the-second-kind}
  \begin{align*}
    k! T(n, k) = \delta^k 0^n
  \end{align*}
\end{lemma}

Meaning that the Knuth's formula

\begin{proposition}[Multifold sums of odd powers in central factorial numbers]
  \label{prop:multifold-sums-of-odd-powers-in-central-factorial-numbers}
  \begin{mdframed}
    For integers $n \geq 1, \; m \geq 1$ and $r\geq0$
    \begin{align*}
      \KnuthRFoldSum{r}{n}{2m-1}
      = \sum_{k=1}^{m} (2k-1)! \binom{n+k-1+r}{2k-1+r} T(2m,2k).
    \end{align*}
  \end{mdframed}
\end{proposition}

originates from Newton's interpolation formula in central
differences~\eqref{prop:newtons-formula-in-central-differences}.

The lemma~\eqref{lem:central-factorial-numbers-of-the-second-kind}
is well discussed
in~\cite{steffensen1927interpolation, carlitz1963divided, riordan1968combinatorial}.

The non-zero central factorial numbers $T(2m,2k)$
is the sequence \href{https://oeis.org/A008957}{\texttt{A008957}}
in the OEIS~\cite{sloane2003line}.

For example,
\begin{align*}
  \KnuthRFoldSum{1}{n}{1} &= \tbinom{n+1}{2} \\
  \KnuthRFoldSum{1}{n}{3} &= 6 \tbinom{n+2}{4} + \tbinom{n+1}{2} \\
  \KnuthRFoldSum{1}{n}{5} &= 120 \tbinom{n+3}{6} + 30 \tbinom{n+2}{4} + \tbinom{n+1}{2} \\
  \KnuthRFoldSum{1}{n}{7} &= 5040 \tbinom{n+4}{8} + 1680 \tbinom{n+3}{6} + 126 \tbinom{n+2}{4} + \tbinom{n+1}{2}
\end{align*}
While multifold sums of odd powers are
\begin{align*}
  \KnuthRFoldSum{r}{n}{1} &= \tbinom{n+1+r}{2+r} \\
  \KnuthRFoldSum{r}{n}{3} &= 6 \tbinom{n+2+r}{4+r} + \tbinom{n+1+r}{2+r} \\
  \KnuthRFoldSum{r}{n}{5} &= 120 \tbinom{n+3+r}{6+r} + 30 \tbinom{n+2+r}{4+r}
  + \tbinom{n+1+r}{2+r} \\
  \KnuthRFoldSum{r}{n}{7} &= 5040 \tbinom{n+4+r}{8+r} + 1680 \tbinom{n+3+r}{6+r}
  + 126 \tbinom{n+2+r}{4+r} + \tbinom{n+1+r}{2+r}
\end{align*}
The coefficients $1, 6, 1, 120, 30, 1,\ldots$
is the sequence \href{https://oeis.org/A303675}{\texttt{A303675}}
in the OEIS~\cite{sloane2003line}.
