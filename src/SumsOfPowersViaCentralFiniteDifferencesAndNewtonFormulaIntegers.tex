\documentclass[10pt]{article}
\textwidth= 5.00in
\textheight= 7.4in
\topmargin = 30pt
\evensidemargin=0pt
\oddsidemargin=55pt
\headsep=17pt
\parskip=.5pt
\parindent=12pt
\font\smallit=cmti10
\font\smalltt=cmtt10
\font\smallrm=cmr9

\usepackage{amssymb,latexsym,amsmath,epsfig,amsthm} %% Add other packages as necessary

\usepackage[
  unicode,                  % allow unicode characters in bookmarks
  pdfpagelabels,            % use page numbers as labels in PDF
  hyperindex,               % add page numbers to index entries as hyperlinks
  colorlinks=true,          % color the text of links instead of boxes
  linkcolor=red,            % color of internal links (sections, equations)
  urlcolor=blue,            % color of URLs
  citecolor=red             % color of citation links
]{hyperref}

\usepackage{geometry} % Page layout and margins
% \geometry{showframe}

\usepackage{booktabs}

\makeatletter

\renewcommand
\section{\@startsection {section}{1}{\z@}
  {-30pt \@plus -1ex \@minus -.2ex}
  {2.3ex \@plus.2ex}
{\normalfont\normalsize\bfseries\boldmath}}

\renewcommand
\subsection{\@startsection{subsection}{2}{\z@}
  {-3.25ex\@plus -1ex \@minus -.2ex}
  {1.5ex \@plus .2ex}
{\normalfont\normalsize\bfseries\boldmath}}

\renewcommand{\@seccntformat}[1]{\csname the#1\endcsname. }

\newcommand\blfootnote[1]{
  \begingroup
  \renewcommand\thefootnote{}\footnote{#1}
  \addtocounter{footnote}{-1}
  \endgroup
}

\makeatother

\newcommand \bernoulli [2][B] {{#1}\sb{#2}}

% factorial notations
\newcommand \centralFactorial [2] {#1^{[#2]}}
\newcommand{\fallingFactorial}[2]{\left(#1\right)_{#2}}
\newcommand{\risingFactorial}[2]{#1^{\left(#2\right)}}

\newcommand{\stirlingii}{\genfrac{\{}{\}}{0pt}{}}
\newcommand{\tstirlingii}[2]{%
  {\textstyle\genfrac\{\}{0pt}{}{#1}{#2}}%
}
\newcommand{\eulerian}[2]{\genfrac{\langle}{\rangle}{0pt}{}{#1}{#2}} % Eulerian
\newcommand{\teulerian}[2]{%
  {\textstyle\genfrac\langle\rangle{0pt}{}{#1}{#2}}%
}

\newcommand{\KnuthRFoldSum}[3]{\Sigma^{#1}\,{#2}^{#3}}

\newcommand{\tsum}{\mathop{\textstyle\sum}}

\newtheorem{theorem}{Theorem}[section]
\newtheorem{corollary}[theorem]{Corollary}
\newtheorem{proposition}[theorem]{Proposition}
\newtheorem{observation}[theorem]{Observation}
\newtheorem{lemma}[theorem]{Lemma}
\newtheorem{claim}[theorem]{Claim}
\newtheorem{example}[theorem]{Example}
\newtheorem{conjecture}[theorem]{Conjecture}
\newtheorem{definition}[theorem]{Definition}
\newtheorem{question}[theorem]{Question}
\newtheorem{remark}[theorem]{Remark}
\newtheorem{assumption}[theorem]{Assumption}

\begin{document}
\begin{figure}
  \vspace*{-62pt}
  %\rightline{\epsfig{file=erdos1.pdf, height=.75in}}
  \vspace{.75in}
\end{figure}
\vspace*{-45pt}
\leftline{\smalltt\#A1} \vskip -12.5pt
\centerline{\smalltt  INTEGERS 26 (2026)}
\vskip 3pt \hrule
\begin{center}
  \uppercase{\bf \boldmath Sums of powers via central finite differences and Newton's formula}
  \blfootnote{DOI: }
  \vskip 20pt
  {\bf Petro Kolosov} \\
  %   {\smallit Department of Mathematics, One University {\tt (outside the U.S)}, City, Country}\\ %Do not include zip codes or postal numbers in authors' affiliations.
  %   {\tt me@math.one.edu}\\ %(optional)
  %   \vskip 10pt
  %   {\bf Author Two}\\
  %   {\smallit Department of Mathematics, University of Two {\tt (inside the U.S)}, City, State}\\ %Do not include zip codes or postal numbers in authors' affiliations.
  %   {\tt you@math.two.edu}\\ %(optional)
\end{center}
\vskip 20pt
\centerline{\smallit Received: , Revised: , Accepted: , Published: } % We will fill in the dates
\vskip 30pt



\centerline{\bf Abstract}
\noindent
In this manuscript we derive formula for multifold sums of powers
using Newton's formula and central differences.




\pagestyle{myheadings}
\markright{\smalltt INTEGERS: 26 (2026)\hfill}
\thispagestyle{empty}
\baselineskip=12.875pt
\vskip 30pt

\section{Introduction and main results}
\begin{proposition}[Newton's series in central differences]
    \label{prop:newtons-series-in-central-differences}
    \begin{align*}
        f(x) &= \sum_{k=0}^{\infty} \frac{\centralFactorial{x}{k}}{k!} \delta^{k} f(0)
    \end{align*}
    where $\delta^{k} f(0)= \sum_{j=0}^{k} (-1)^{j} \binom{k}{j} f\left(\frac{k}{2} - j\right)$
    is central finite difference in zero, and $\centralFactorial{x}{k} = n \left( n + \frac{k}{2} -1 \right)\left( n + \frac{k}{2} -2 \right) \cdots \left( n - \frac{k}{2} +1 \right)$ is central factorial.
\end{proposition}

\begin{lemma}[Central factorial]
    \begin{align*}
        \centralFactorial{n}{k}
        = n \left( n + \frac{k}{2} -1 \right)\left( n + \frac{k}{2} -2 \right) \cdots \left( n - \frac{k}{2} +1 \right)
        = n \prod_{j=1}^{k-1} \left( n + \frac{k}{2} -j \right)
    \end{align*}
\end{lemma}
We observe that central factorials are closely related to falling factorials $\fallingFactorial{x}{n} = x(x-1)(x-2)(x-3)\cdots(x-n+1)=\prod_{k=0}^{n-1}(x-k)$.
Therefore,
\begin{align*}
    \centralFactorial{n}{k} = n \fallingFactorial{n+\frac{k}{2}-1}{k-1}
\end{align*}
To derive formula for multifold sums of powers, we follow the strategy to express the Newton's formula [eq ref] in terms
of binomial coefficients, then to reach closed forms of column sum of binomial coefficients by means of hockey stick pattern.
Therefore,
\begin{proposition}
    For $k \geq 1$
    \begin{align*}
        \frac{\centralFactorial{n}{k}}{k!}
        &= \frac{n}{k!}\fallingFactorial{n+\frac{k}{2}-1}{k-1}
        = \frac{n}{k (k-1)!} \fallingFactorial{n+\frac{k}{2}-1}{k-1}
        = \frac{n}{k} \binom{n+\frac{k}{2}-1}{k-1}
    \end{align*}
    \begin{proof}
        The identity above is true because $\frac{\fallingFactorial{x}{n}}{n!} = \binom{x}{n}$.
    \end{proof}
\end{proposition}

Which yields Newton's formula for powers, in terms of central differences.
For positive integers $n \geq 1$ and $m \geq 1$
\begin{align*}
    n^m = \sum_{k=1}^{m} \frac{n}{k} \binom{n+\frac{k}{2}-1}{k-1} \delta^{k} 0^m
\end{align*}

In the proposition above,
we start the summation from $k=1$ to avoid division by zero in $\frac{n}{k}$.
It is a valid trick, because the central difference $\delta^{k} 0^n$ is zero for all $n \geq 1$ and $k=0$.

By factoring out and simplifying the term $n$, we get
\begin{align*}
    n^{m-1} = \sum_{k=1}^{m} \frac{1}{k} \binom{n+\frac{k}{2}-1}{k-1} \delta^{k} 0^m
\end{align*}
By reindexing the sum yields
\begin{proposition} [Newton's series for power in zero]
    \begin{align*}
        n^{m} = \sum_{k=0}^{m+1} \frac{1}{k+1} \binom{n+\frac{k+1}{2}-1}{k} \delta^{k+1} 0^{m+1}
    \end{align*}
\end{proposition}
Thus, formula for ordinary sums of powers follows
\begin{proposition} [Ordinary sums of powers]
    \begin{align*}
        \KnuthRFoldSum{1}{n}{m} = \sum_{k=0}^{m+1} \frac{1}{k+1} \binom{n+\frac{k+1}{2}}{k+1} \delta^{k+1} 0^{m+1}
    \end{align*}
    \begin{proof}
        We have
        $\KnuthRFoldSum{1}{n}{m} = \sum_{k=0}^{m} \frac{1}{k+1} \delta^{k+1} 0^{m+1} \sum_{j=1}^{n} \binom{j+\frac{k+1}{2}-1}{k}$.
        By hockey stick identity $\sum_{j=1}^{n} \binom{j+\frac{k+1}{2}-1}{k} = \binom{n+\frac{k+1}{2}}{k+1}$.
        Thus, the claim follows.
    \end{proof}
\end{proposition}
Continuing similarly, we get formula for multifold sums of powers
\begin{theorem}[Multifold sums of powers]
    \begin{mdframed}
        \begin{align*}
            \KnuthRFoldSum{r}{n}{m} = \sum_{k=0}^{m+1} \frac{1}{k+1} \binom{n+\frac{k+1}{2}-1+r}{k+r} \delta^{k+1} 0^{m+1}
        \end{align*}
    \end{mdframed}
\end{theorem}
Additionally, the formula for multifold sums of powers can be expressed in terms of central factorial numbers of the second kind [references].
In Riordan notation,

\begin{lemma} [Central factorial numbers]
    \begin{align*}
        T(n, k) = \frac{\delta^k 0^n}{k!}
    \end{align*}
\end{lemma}
Note that central factorial numbers of the second kind $T(n,k)$ are non-zero
only for pairs $(n,k)$ such that $n-k$ is even.
Meaning that $T(2n,2k)$ is always non-zero.
The triangle of central factorial numbers $T(2n,2k)$ is the sequence [ID] in the OEIS~\cite{sloane2003line}.

\begin{proposition}[Multifold sums of powers via central factorial numbers]
    \begin{mdframed}
        \begin{align*}
            \KnuthRFoldSum{r}{n}{m} = \sum_{k=0}^{m} k! \binom{n+\frac{k+1}{2}-1+r}{k+r} T(m+1, k+1)
        \end{align*}
    \end{mdframed}
\end{proposition}




\section*{Conclusions}
In this manuscript, we derived formula for multifold sums of powers
using Newton's formula in central differences, combined with
hockey-stick identity for binomial coefficients.
Additionally, we shown that the famous Knuth's formula for multifold sums of
powers~\cite{knuth1993johann} originates from
Newton's formula in central differences.


\bibliographystyle{unsrt}
\bibliography{SumsOfPowersViaCentralFiniteDifferencesAndNewtonFormulaIntegers.bib}

\clearpage

\section{Mathematica programs} \label{sec:mathematica}
Use the \textit{Mathematica} package~\cite{GitHubSourceCode} to validate the results
\begin{center}
  \renewcommand{\arraystretch}{1.3}
  \begin{tabular}{ll}
    \toprule
    \textbf{Mathematica Function} & \textbf{Validates / Prints}  \\
    \midrule
    \texttt{MultifoldSumOfPowersRecurrence[r, n, m]}
    & Computes $\KnuthRFoldSum{r}{n}{m}$ \\
    \texttt{ValidateCentralFactorialsInTermsOfFalling[10]}
    & Validates Proposition~\eqref{prop:central-factorials-in-terms-of-falling} \\
    \texttt{ValidateBinomialFormOfCentralFactorials[10]}
    & Validates Proposition~\eqref{prop:binomial-form-of-central-factorials} \\
    \texttt{ValidateNewtonsFormulaForPowersInZero[20]}
    & Validates Proposition~\eqref{prop:newtons-formula-for-powers-in-zero} \\
    \texttt{ValidateOrdinarySumsOfOddPowersInCentralDifferences[20]}
    & Validates Prop.~\eqref{prop:ordinary-sums-of-odd-powers-in-central-differences} \\
    \texttt{ValidateMultifoldSumsOfOddPowersInCentralDifferences[5]}
    & Validates Thm.~\eqref{theorem:multifold-sums-of-odd-powers-in-central-differences} \\
    \texttt{ValidateNewtonsFormulaForPowers[10]}
    & Validates Prop.~\eqref{prop:newtons-formula-for-powers} \\
    \texttt{ValidatePowersInCentralBinomialForm[10]}
    & Validates Prop.~\eqref{prop:powers-in-central-binomial-form} \\
    \bottomrule
  \end{tabular}
\end{center}


% \begin{thebibliography}{1}\footnotesize

%   \bibitem{A} J. Author, My first math paper, {\it Integers} {\bf 25} (2020), \#A32, 8 pp.

%   \bibitem{BS} S. Biswas and N. Saikia, Arithmetic properties of the coefficients of some mock theta functions, {\it Integers} {\bf 25} (2025), \#A72, 15 pp.

%   \bibitem{CD} I. Can and N. Do, Proof of existence, {\it J. Math. Stuff} {\bf 17} (2000), 19--23.

%   \bibitem{C} D. B. Cooper, {\it How to Make Money}, Ph.D. thesis, Skyfall University, 1971.

%   \bibitem{DD} J. Deer and K. Doe, On the history of mathematics, {\it J. of the World} {\bf 52} (3) (1999), 123--135.

%   \bibitem{EZ} R. Evans and D. Zager, Math in the future, preprint, {\tt arXiv:2525.31415}.

%   \bibitem{JSV} A. Jones, L. Smith, and C. Vector, {\it The Theory of Everything}, Publishing Company, New York, 1987.

%   \bibitem{OEIS} OEIS Foundation Inc.,  The On-Line Encyclopedia of Integer Sequences, {\tt https://oeis.org}.

%   \bibitem{RW} A. Reid and B. Wright, Important number theory result, in {\it The Big Math Book}, Publishing Company, Boston, MA, 1975, 123--135.

%   \bibitem{S} A. Smith, Math to be published, {\it J. Future Math.}, to appear.

% \end{thebibliography}


\end{document}
