\documentclass[12pt,letterpaper,oneside,reqno]{amsart}
\usepackage{amsfonts}
\usepackage{amsmath}
\usepackage{amssymb}
\usepackage{amsthm}
\usepackage{float}
\usepackage{mathrsfs}
\usepackage{colonequals}
\usepackage[font=small,labelfont=bf]{caption}
\usepackage[unicode,pdfpagelabels,hyperindex,colorlinks=true,linkcolor=red,urlcolor=blue,citecolor=red]{hyperref}
\usepackage{graphicx}
\emergencystretch=1em
\usepackage{array}
\usepackage{enumitem}
\usepackage{etoolbox}
\usepackage{physics}
\usepackage{booktabs}
\usepackage{url}
\usepackage{mdframed}

% margins and layout
\linespread{1.7}
\usepackage[left=1in,right=1in,bottom=1in,top=1in]{geometry}
\apptocmd{\sloppy}{\hbadness 10000\relax}{}{}
\raggedbottom

\newcommand \coeffA [3][A] {{\mathbf{#1}} \sb{#2,#3}}
\newcommand \polynomialP [4][P]{{\mathbf{#1}}\sp{#2} \sb{#3}(#4)}
\newcommand \bernoulli [2][B] {{#1}\sb{#2}}

%\newcommand \anglePower [2]{\langle #1 \rangle \sp{#2}}
%\newcommand \curvePower [2]{\{#1\}\sp{#2}}

% Falling, rising and central factorials
\newcommand{\fallingFactorial}[2]{\left(#1\right)_{#2}}
\newcommand{\risingFactorial}[2]{#1^{\left(#2\right)}}
\newcommand \centralFactorial [2] {#1^{[#2]}}

\newcommand{\stirlingii}{\genfrac{\{}{\}}{0pt}{}}
\newcommand{\eulerian}[2]{\genfrac{\langle}{\rangle}{0pt}{}{#1}{#2}} % Eulerian

\newcommand{\KnuthRFoldSum}[3]{\Sigma^{#1}\,{#2}^{#3}}


\newtheorem{theorem}{Theorem}[section]
\newtheorem{corollary}[theorem]{Corollary}
\newtheorem{proposition}[theorem]{Proposition}
\newtheorem{observation}[theorem]{Observation}
\newtheorem{lemma}[theorem]{Lemma}
\newtheorem{claim}[theorem]{Claim}
\newtheorem{example}[theorem]{Example}
\newtheorem{conjecture}[theorem]{Conjecture}
\newtheorem{definition}[theorem]{Definition}
\newtheorem{question}[theorem]{Question}
\newtheorem{remark}[theorem]{Remark}
\newtheorem{assumption}[theorem]{Assumption}

%\numberwithin{equation}{section}

\title[Sums of powers via central finite differences and Newton's formula]
{Sums of powers via central finite differences and Newton's formula}
\author[Petro Kolosov]{Petro Kolosov}
\date{\today}

% 05A19 — Combinatorial identities
% 05A10 — Factorials, binomial coefficients, combinatorial functions
% 41A15 — Spline approximation
% 11B68 — Bernoulli and Euler numbers and polynomials
% 11B73 — Bell and Stirling numbers
% 11B83 — Special sequences and polynomials
\subjclass[2010]{05A19, 05A10, 11B73, 11B83}

% metadata
%\email{kolosovp94@gmail.com}
%\address{Software Developer, DevOps Engineer}
%\urladdr{https://kolosovpetro.github.io}
%\keywords{Sums of powers,
%    Newton's interpolation formula,
%    Finite differences,
%    Binomial coefficients,
%    Faulhaber's formula,
%    Bernoulli numbers,
%    Bernoulli polynomials,
%    Interpolation,
%    Discrete convolution,
%    Combinatorics,
%    Polynomial identities,
%    Central factorial numbers,
%    Stirling numbers,
%    Eulerian numbers,
%    Worpitzky identity,
%    Pascal's triangle,
%    OEIS}
%\hypersetup{
%    pdftitle={Sums of powers via central finite differences and Newton's formula},
%    pdfproducer={LaTeX},
%    pdfcreator={pdflatex},
%    pdfauthor={Petro Kolosov},
%    pdfsubject={Sums of powers via central finite differences and Newton's formula},
%    pdfkeywords={Sums of powers,
%    Power sums,
%    Polynomials,
%    Power function,
%    Polynomial identities,
%    Newton's interpolation formula,
%    Finite differences,
%    Interpolation,
%    Binomial coefficients,
%    Multinomial coefficients,
%    Binomial theorem,
%    Faulhaber's formula,
%    Bernoulli numbers,
%    Bernoulli polynomials,
%    Combinatorics,
%    Central factorial numbers,
%    Stirling numbers,
%    Eulerian numbers,
%    Worpitzky identity,
%    Pascal's triangle,
%    OEIS}
%}

\begin{document}

    \begin{abstract}
        In this manuscript we derive formula for multifold sums of powers
using Newton's formula and central differences.

    \end{abstract}

    \maketitle

    \addcontentsline{toc}{section}{Abstract}

%    \tableofcontents


    \section{Introduction and main results} \label{sec:introduction}
    \begin{proposition}[Newton's series in central differences]
    \label{prop:newtons-series-in-central-differences}
    \begin{align*}
        f(x) &= \sum_{k=0}^{\infty} \frac{\centralFactorial{x}{k}}{k!} \delta^{k} f(0)
    \end{align*}
    where $\delta^{k} f(0)= \sum_{j=0}^{k} (-1)^{j} \binom{k}{j} f\left(\frac{k}{2} - j\right)$
    is central finite difference in zero, and $\centralFactorial{x}{k} = n \left( n + \frac{k}{2} -1 \right)\left( n + \frac{k}{2} -2 \right) \cdots \left( n - \frac{k}{2} +1 \right)$ is central factorial.
\end{proposition}

\begin{lemma}[Central factorial]
    \begin{align*}
        \centralFactorial{n}{k}
        = n \left( n + \frac{k}{2} -1 \right)\left( n + \frac{k}{2} -2 \right) \cdots \left( n - \frac{k}{2} +1 \right)
        = n \prod_{j=1}^{k-1} \left( n + \frac{k}{2} -j \right)
    \end{align*}
\end{lemma}
We observe that central factorials are closely related to falling factorials $\fallingFactorial{x}{n} = x(x-1)(x-2)(x-3)\cdots(x-n+1)=\prod_{k=0}^{n-1}(x-k)$.
Therefore,
\begin{align*}
    \centralFactorial{n}{k} = n \fallingFactorial{n+\frac{k}{2}-1}{k-1}
\end{align*}
To derive formula for multifold sums of powers, we follow the strategy to express the Newton's formula [eq ref] in terms
of binomial coefficients, then to reach closed forms of column sum of binomial coefficients by means of hockey stick pattern.
Therefore,
\begin{proposition}
    For $k \geq 1$
    \begin{align*}
        \frac{\centralFactorial{n}{k}}{k!}
        &= \frac{n}{k!}\fallingFactorial{n+\frac{k}{2}-1}{k-1}
        = \frac{n}{k (k-1)!} \fallingFactorial{n+\frac{k}{2}-1}{k-1}
        = \frac{n}{k} \binom{n+\frac{k}{2}-1}{k-1}
    \end{align*}
    \begin{proof}
        The identity above is true because $\frac{\fallingFactorial{x}{n}}{n!} = \binom{x}{n}$.
    \end{proof}
\end{proposition}

Which yields Newton's formula for powers, in terms of central differences.
For positive integers $n \geq 1$ and $m \geq 1$
\begin{align*}
    n^m = \sum_{k=1}^{m} \frac{n}{k} \binom{n+\frac{k}{2}-1}{k-1} \delta^{k} 0^m
\end{align*}

In the proposition above,
we start the summation from $k=1$ to avoid division by zero in $\frac{n}{k}$.
It is a valid trick, because the central difference $\delta^{k} 0^n$ is zero for all $n \geq 1$ and $k=0$.

By factoring out and simplifying the term $n$, we get
\begin{align*}
    n^{m-1} = \sum_{k=1}^{m} \frac{1}{k} \binom{n+\frac{k}{2}-1}{k-1} \delta^{k} 0^m
\end{align*}
By reindexing the sum yields
\begin{proposition} [Newton's series for power in zero]
    \begin{align*}
        n^{m} = \sum_{k=0}^{m+1} \frac{1}{k+1} \binom{n+\frac{k+1}{2}-1}{k} \delta^{k+1} 0^{m+1}
    \end{align*}
\end{proposition}
Thus, formula for ordinary sums of powers follows
\begin{proposition} [Ordinary sums of powers]
    \begin{align*}
        \KnuthRFoldSum{1}{n}{m} = \sum_{k=0}^{m+1} \frac{1}{k+1} \binom{n+\frac{k+1}{2}}{k+1} \delta^{k+1} 0^{m+1}
    \end{align*}
    \begin{proof}
        We have
        $\KnuthRFoldSum{1}{n}{m} = \sum_{k=0}^{m} \frac{1}{k+1} \delta^{k+1} 0^{m+1} \sum_{j=1}^{n} \binom{j+\frac{k+1}{2}-1}{k}$.
        By hockey stick identity $\sum_{j=1}^{n} \binom{j+\frac{k+1}{2}-1}{k} = \binom{n+\frac{k+1}{2}}{k+1}$.
        Thus, the claim follows.
    \end{proof}
\end{proposition}
Continuing similarly, we get formula for multifold sums of powers
\begin{theorem}[Multifold sums of powers]
    \begin{mdframed}
        \begin{align*}
            \KnuthRFoldSum{r}{n}{m} = \sum_{k=0}^{m+1} \frac{1}{k+1} \binom{n+\frac{k+1}{2}-1+r}{k+r} \delta^{k+1} 0^{m+1}
        \end{align*}
    \end{mdframed}
\end{theorem}
Additionally, the formula for multifold sums of powers can be expressed in terms of central factorial numbers of the second kind [references].
In Riordan notation,

\begin{lemma} [Central factorial numbers]
    \begin{align*}
        T(n, k) = \frac{\delta^k 0^n}{k!}
    \end{align*}
\end{lemma}
Note that central factorial numbers of the second kind $T(n,k)$ are non-zero
only for pairs $(n,k)$ such that $n-k$ is even.
Meaning that $T(2n,2k)$ is always non-zero.
The triangle of central factorial numbers $T(2n,2k)$ is the sequence [ID] in the OEIS~\cite{sloane2003line}.

\begin{proposition}[Multifold sums of powers via central factorial numbers]
    \begin{mdframed}
        \begin{align*}
            \KnuthRFoldSum{r}{n}{m} = \sum_{k=0}^{m} k! \binom{n+\frac{k+1}{2}-1+r}{k+r} T(m+1, k+1)
        \end{align*}
    \end{mdframed}
\end{proposition}




    \section{Proof of Knuth's formula}\label{sec:proof-of-knuth's-formula}
    \begin{proposition}[Knuth's formula for Multifold sums of odd powers]
    \label{prop:knuth-sums-of-odd-powers}
    \begin{align*}
        \KnuthRFoldSum{r}{n}{2m-1} &= \sum_{k=1}^{m} (2k-1)! \binom{n+k-1+r}{2k-1+r} T(2m, 2k)
    \end{align*}
    where $T(n, k)$ is central factorial number $T(n, k) = \frac{1}{k!} \delta^k 0^n$ and $\delta^k 0^n$
    is the central finite difference of power in zero.
    \begin{proof}
        Consider the Riordan's power identity, see [cite]
        \begin{lemma} [Riordan's power identity]
            \begin{align*}
                n^m = \sum_{k=1}^{m} T(m,k) \centralFactorial{n}{k}
            \end{align*}
            where $\centralFactorial{n}{k}$ is central factorial [eq ref].
        \end{lemma}
        It is easy to see that Riordan's power identity is a direct consequence of Newton's series for central difference [eq ref],
        with $T(m,k)=\frac{1}{k!} \delta^k 0^m$, where $\delta^k 0^m$ is central difference of power in zero.
        Now we notice that $T(m,k) =0$ whether $m-k$ is odd.
        Thus,
        \begin{align*}
            n^{2m} = \sum_{k=1}^{2m} T(2m,k) \centralFactorial{n}{k}
        \end{align*}
        We allow $k$ to run over the integers $k=2,4,6,\ldots, 2m$ to maintain the condition $T(2m,k) \neq 0$,
        hence
        \begin{align*}
            n^{2m} = \sum_{k=1}^{m} T(2m,2k) \centralFactorial{n}{2k}
        \end{align*}
        By expressing the central factorials $\centralFactorial{n}{2k}$ in terms of falling factorials
        \begin{align*}
            \centralFactorial{n}{2k}
            = n \fallingFactorial{n+k-1}{2k-1}
        \end{align*}
        Yields
        \begin{align*}
            n^{2m} = \sum_{k=1}^{m} T(2m,2k) n \fallingFactorial{n+k-1}{2k-1}
        \end{align*}
        By dividing by $n$ an applying the identity $\frac{\fallingFactorial{x}{n}}{n!} = \binom{x}{n}$, we get
        \begin{align*}
            n^{2m-1} = \sum_{k=1}^{m} (2k-1)! \binom{n+k-1}{2k-1} T(2m,2k)
        \end{align*}
        which is the base identity for odd powers.

        Now, the ordinary sum of odd powers is
        \begin{align*}
            \KnuthRFoldSum{1}{n}{2m-1} = \sum_{k=1}^{m} (2k-1)! T(2m,2k) \sum_{j=1}^{n} \binom{j+k-1}{2k-1}
        \end{align*}
        By hockey-stick identity $\sum_{j=1}^{n} \binom{j+k-1}{2k-1} = \binom{n+k}{2k}$, thus
        \begin{align*}
            \KnuthRFoldSum{1}{n}{2m-1} = \sum_{k=1}^{m} (2k-1)! T(2m,2k) \binom{n+k}{2k}
        \end{align*}
        By induction yields
        \begin{align*}
            \KnuthRFoldSum{r}{n}{2m-1} &= \sum_{k=1}^{m} (2k-1)! \binom{n+k-1+r}{2k-1+r} T(2m, 2k)
        \end{align*}
    \end{proof}
\end{proposition}
We can compare the Knuth's formula and special case of [eqref] for $m \rightarrow 2m-1$, that is
\begin{corollary}
    \begin{align*}
        \KnuthRFoldSum{r}{n}{2m-1} = \sum_{k=0}^{m} k! \binom{n+\frac{k+1}{2}-1+r}{k+r} T(2m, 2k)
    \end{align*}
\end{corollary}



    \section{Conclusions}\label{sec:conclusions}
    In this manuscript, we derived formula for multifold sums of powers
using Newton's formula in central differences, combined with
hockey-stick identity for binomial coefficients.
Additionally, we shown that the famous Knuth's formula for multifold sums of
powers~\cite{knuth1993johann} originates from
Newton's formula in central differences.



    \section{Acknowledgements}\label{sec:acknowledgements}
    \input{sections/acknowledgements}

    \bibliographystyle{unsrt}
    \bibliography{SumsOfPowersViaCentralFiniteDifferencesAndNewtonFormula}

    {\large\textbf{Metadata}}
\begin{itemize}
  \item \textbf{Initial release date:} January 3, 2026.
  \item \textbf{Current release date:} \today.
  \item \textbf{Version:} \input{metadata/version}.
  \item \textbf{MSC2010:} 05A19, 05A10, 11B73, 11B83.
  \item \textbf{Keywords:} \input{metadata/footer-keywords.tex}
  \item \textbf{License:} This work is licensed under a \href{https://creativecommons.org/licenses/by/4.0/}{\texttt{CC BY 4.0 License}}.
  \item \textbf{DOI:} \href{https://doi.org/10.5281/zenodo.18096789}{\texttt{https://doi.org/10.5281/zenodo.18096789}}
  \item \textbf{Web Version:} \href{https://kolosovpetro.github.io/sums-of-powers-central-differences/}{\texttt{kolosovpetro.github.io/sums-of-powers-central-differences/}}
  \item \textbf{Sources:} \href{https://github.com/kolosovpetro/SumsOfPowersViaCentralFiniteDifferencesAndNewtonFormula}{\texttt{github.com/kolosovpetro/SumsOfPowersViaCentralFiniteDifferencesAndNewtonFormula}}
  \item \textbf{ORCID:} \href{https://orcid.org/0000-0002-6544-8880}{\texttt{0000-0002-6544-8880}}
  \item \textbf{Email:} \href{mailto:kolosovp94@gmail.com}{\texttt{kolosovp94@gmail.com}}
\end{itemize}


\end{document}
